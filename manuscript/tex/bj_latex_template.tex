% Template for BJ paper in LaTeX (Jan Hrabe, NKI, 05/16/2005)
%
% To compile into a document, run
% latex bj_latex_template
% bibtex bj_latex_template
% latex bj_latex_template (run 2-3 times repeatedly)
% dvips bj_latex_template.dvi
%
% or replace the latex command by the pdflatex command in the lines above to
% generate a PDF file and use acroread or xpdf for viewing and
% printing instead of the postscript generating program dvips

% Use standard article document class with slightly bigger font and 
% a separate title page
\documentclass[11pt,titlepage]{article}

% Packages to load (all standard on a modern LaTeX system on Linux)

% Make doublespaced ugly typography required for mysterious 
% reasons by most journals - comment out for normal output
\usepackage{setspace} 
\doublespacing

% AMS-Math package to have nice multi-line equations and other goodies
\usepackage{amsmath}

% Show labels for easy orientation, comment out for final version
% \usepackage{showlabels}

% EPS/PDF graphics
% Place figures in the document directory in both the EPS and PDF
% formats, e.g., fig_1.eps and fig_1.pdf. Use the includegraphics
% command without file extension, e.g. \includegraphics*[width=3.25in]{fig_1}
% The pdflatex or latex programs then work automagically with the 
% appropriate formats.  EPS figures can be converted to PDF using
% the epstopdf program present on most Linux disributions
\usepackage{graphicx}

% Citation style in the text: numbers in parenthesis, sorted by their
% order in the list of references.
% Uses a range if possible: (1-3), not (1,2,3)
\usepackage[round,numbers,sort&compress]{natbib} 
% Bibliography style (requires the style file biophysj.bst in the 
% document directory)
\bibliographystyle{biophysj}
% Numbering style in the list of references: a number followed by a period
\renewcommand{\bibnumfmt}[1]{#1.}

% Examples of special definitions (amsmath package required)
\newcommand{\erf}{\operatorname{erf}}        % error function
\newcommand{\erfc}{\operatorname{erfc}}      % complementary error function
\newcommand{\BibTeX}{\textsc{Bib}\TeX}       % corect BibTeX appearance

\title{Paper title in full}

\author{John~Doe\thanks{
           Corresponding author.  Address: 
           Department name,
	   Institute name,
	   First address line,
	   City, State~00000, U.S.A.,
	   Tel.:~(000)000-0000, Fax:~(000)000-0000} \\
	Department name, \\
	Institute name, City, State 
	\and Jenny Doe \\
	Department name, \\
	Institute name, City, State}

% Revision date - uncomment to exclude date in the final version
\date{}

% Running head
\pagestyle{myheadings}
\markright{Short paper title}

% We are done with the headers, the actual document starts here
\begin{document}

% generate the title page from the info in the headers above
\maketitle

% 200 words max Abstract
\abstract{Abstract text here.

\emph{Key words:} keyword 1; keyword 2; keyword 3; keyword 4; 
keyword 5; keyword 6}

% New page
\clearpage

\section*{Introduction}

Citations are made using the \emph{citep} command, e.g., one paper
\citep{el-Kareh_etal93} or more papers
\citep{el-Kareh_etal93,Chen_Nicholson00}.  It works fine with a single
author, two authors, or more.  Books are cited in the same way
\citep{Callaghan91}, book chapters as well \citep{Stiles_Bartol01},
including a chapter in press \citep{Stiles_etal04}.  Abstracts can be
handled too \citep{Tao_etal02}.  Pages can be included in a citation
\citep[pp.~12--18]{Callaghan91}.  Citations in a text form (i.e., author 
name followed by the usual reference number in parenthesis) can be 
done using the \emph{citet} command, e.g., ``an approach used 
by \citet{el-Kareh_etal93}''.

\section*{Theory}

Text can reference Eq.~\ref{eqn:symmetry}
\begin{equation} \label{eqn:symmetry}
   \Phi(\vec{r}) = \Phi(-\vec{r})
\end{equation}
anywhere, as long as it is numbered.

\section*{Methods}

\subsection*{Subsection}

\subsubsection*{Subsubsection}

Further text subdivisions can be made with the \emph{subsection} and
\emph{subsubsection} commands.

\section*{Results}

References to figures use the \emph{ref} command, similarly to equation
references.  See Fig.~\ref{fig:result_fig}.

\section*{Discussion}

The source file for this document is called
\emph{bj\_latex\_template.tex}.  Apart from this \LaTeX\ file, you
will also need the bibliography file, the \BibTeX\ style file, and the
EPS and PDF figure files.

See the bibliography file \emph{bj\_bibtex\_template.bib} for the
literature data.  It was mostly generated from the saved
text-formatted PubMed entries using the \emph{med2bib} program and
edited by the \emph{tkbibtex} or directly in the \emph{emacs} editor.

The \emph{biophysj.bst} file is a \BibTeX\ style file that contains
information about the format required by Biophysical Journal for the
list of references.

Figure file \emph{fig\_1.eps} was generated by \emph{xfig} program and
converted to PDF by the \emph{epstopdf} program.  Most data plotting
is easily accomplished by the \emph{gnuplot} program.

% Compile and format the bibliography (bj_bibtex_template.bib BibTeX
% file must be present in the document directory)
\bibliography{bj_bibtex_template}

% Figure legends
\clearpage
\section*{Figure Legends}
\subsubsection*{Figure~\ref{fig:result_fig}.}
Figure legend here.

% Figures, one per page (fig_1.eps and fig_1.pdf files must be present
% in the document directory)
\clearpage
\begin{figure}
   \begin{center}
      \includegraphics*[width=3.25in]{fig_1}
      \caption{}
      \label{fig:result_fig}
   \end{center}
\end{figure}

% closing statement, nothing below matters
\end{document}
