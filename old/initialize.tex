\documentclass{article}

\input{C:/D/working_copies/neur_ca_imag/trunk/my_latex_defs}

\title{Initializing Calcium Filter}

\author{Joshua Vogelstein}

\begin{document}

\section{Initializing the initializer}

Given a set of observations, $\{F_t\}_1^T$, we can initialize an intializer by first computing the numerical derivative of the signal and then thresholding.  The level of the threshold is set by the expected number of spikes, i.e., if the duration of the observations is $\tau$ and the estimated firing rate of the neuron is $fr$, then the threshold is $thr = fr \times \tau$.  Thus, we have:

\begin{equation}
\widehat{n}_t = 
\begin{cases}
1 & \text{if } F_t - F_{t-1} > thr\\
0 & \text{otherwise}
\end{cases}
\end{equation}

yielding an intial guess of spike times, $\{\widehat{n}_t\}_1^N$. Having an initial guess of when the neuron spikes, we can get an initial guess of the calcium filter.  Assume that the kernel a $1 \times p$ column vector, we then define the intial guess of the kernel as

\begin{equation}
K(s) =  \frac{1}{N}\sum_{n_t: n_t =1} F_{t+s} \qquad s=0,\ldots,p-1
\end{equation}

We then fit this kernel to an exponential with peak $A$ and time constant $\tau_c$:

\begin{align}
\widehat{A} &= \argmax_t F_t - F_{t-1}\\
\widehat{\tau}_c &= \max_{\tau_c} \norm{\{A e^{-t/\tau_c} - K(t)\}_{t=1}^T}^2
\end{align}


\section{Projection Pursuit Regression}

Given a kernel, $K$, operating on the spike train yielding the calicum concentration, we are trying to find the most likely spike train, i.e.,

\begin{equation}
\{\widehat{n}_t\}=\max_{\substack{\{n_t\}\\ n_t \in \{0,1\}}}\sum_t \left( \left(F_t - \sum_{s=1}^p K_{s} n_{t-s}\right)^2 + \lambda n_t\right)
\end{equation}

We adopt an iterative procedure where we add one spike at a time.  Thus the goal with each iteration is to find the location of a spike that most increases the likelihood.  The algorithm converges when adding another spike necessarily decreases the likelihood.  To find the location of the spike, we have:

\begin{align}
t^{(1)} &= \max_{t'} \sum_t \left( \left(F_t - \sum_{s=1}^p K_s n_{t'-s}\right)^2 + \lambda n_{t'}\right) \\
&= \max_{t'} \sum_t \left( F_t^2 - 2 F_t  \sum_{s=1}^p K_s n_{t'-s} + ( \sum_{s=1}^p K_s n_{t'-s})^2  + \lambda n_{t'}\right) \\
&= \max_{t'} \sum_t \left(- 2 F_t  \sum_{s=1}^p K_s n_{t'-s}\right)\\
&= \min_{t'} \sum_t F_t \sum_{s=1}^p K_s n_{t'-s}
\end{align}

\noindent where the second equality follows from expanding the square, the second comes from dropping terms that are constant as a function of $t'$.  Thus, the above maximization\footnote{min?} can be solved simply by computing a cross-correlation between $\{F_t\}$ and  $\{\sum_{s=1}^p K_s n_{t'-s}\}$ and finding the max, i.e.,

\begin{align}
R_{KF}(t) &= \sum_{m=0}^T F_{t+m} \sum_{s=1}^p K_s n_{t'-s+m}\\
n^{(1)} &= \argmax_t R_{KF} (t) 
\end{align}

Having chosen the location of the first spike, we can proceed to choose the location of the next spike.  First, we recursively define the residual as $r_t = r_t - \sum_{s=1}^p K_s n_{t^{(i)}-s}$ where $t^{(i)}$ corresponds to the time of the most recently added spike to the spike train.  


\end{document}
