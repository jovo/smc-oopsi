Recently, great strides in technology and ingenuity have led to the development of an assortment of calcium sensitive fluorophores for use in a wide array of neural substrates\cite{Tsien83, DenkWebb90, MiyawakiTsien97, StosiekKonnerth03, SvobodaYasuda06}.  Both genetically engineered proteins\cite{KleinfeldGriesbeck05} and organic dyes\cite{HelmchenWaters02} have yielded considerable insight into neurobiological questions, with other technologies, such as quantum dots, emerging\cite{SlotkinHaydar07,CuiChu07} (see \cite{GiepmansTsien06} for a comprehensive review of calcium sensors).  These strategies are becoming increasingly popular, as they enable  simultaneously imaging populations of neurons both \emph{in vitro}\cite{IkegayaYuste04} and \emph{in vivo}\cite{NiellSmith05, OhkiReid05, OhkiReid06, YaksiFriedrich07, NagayamaChen07, SatoSvoboda07,RootWang07}, allowing experimentalists to ask previously intractable questions related to population coding.  However, the fluorescence signals available from these technologies are not a panacea; but rather, typically suffer from two problems.  First, the rate of data acquisition tends to be on the order of tens to hundreds of milliseconds\cite{MajewskaYuste00, TsaiKleinfeld02}, even though the neurons operate on time scales of individual milliseconds.  While new video-rate\cite{FanEllisman99, NguyenParker01,RoordaMiesenbock04} or faster\cite{SalomeBourdieu06, IyerSaggau06} scan rates promise to rectify this problem, they will also exacerbate the second problem --- low signal-to-noise ratio --- as the dwell time per pixel necessarily decreases as scan rates increase. Thus, computational tools for analyzing this kind of data must be developed.  In particular, two questions are of general interest: (i) precisely when do the observable neurons spike, and (ii) what caused those spikes.

To our knowledge, previous efforts have addressed one or the other of these two problems, but not both.  In 2005, Kerr \emph{et al.}\cite{KerrHelmchen05} used a custom template-matching algorithm to detect the presence of single spikes, effectively identifying in which frame spikes occurred for sparsely spiking data sets, but could only achieve a precision commiserate with the imaging frame rate, and multiple spikes per frame were not considered. The following year, Yaksi and Friedrich\cite{YaksiFriedrich06} developed a linear smoothing convolution kernel that effectively inferred the time-varying firing rate underlying the observed calcium transients, but could not obtain precise firing times, and required fitting the calcium parameters by doing tedious electrophysiological recordings in a few cells, and then assuming those parameters hold for all the other neurons. At about the same time, Ramdya \emph{et al.}\cite{RamdyaEngert06} demonstrated that they could learn the same parameters of a linear filter driving neural activity from either calcium data or voltage-clamp, but did not address the issue of resolving spikes. More recently, Sato \emph{et al.} designed a clustering algorithm to determine whether whisker stimulation successfully induced a spike, but this approach is limited to resolving the presence or absence of a single spike\cite{SatoSvoboda07}.

The present work differs from previous related work in several key aspects.  We developed a technique based on a first-principles approach of how neurons behave and how their corresponding signals are measured, as opposed to a brute-force application of standard linear signal processing tools.  In particular, we assume a realistic model underlying the neural activity, providing a prior predicting how neurons respond to external sensorimotor covariates.  Further, our model incorporates the fact that the actual data is both noisy and intermittent.  This framework enables us to address, under a single umbrella, both precisely when spikes occurred, and what led to those spikes.  And because our approach is completely unsupervised, it obviates the need to train using simultaneously acquired intracellular electrophysiology and calcium data, which is often a considerable undertaking.  These differences enable our algorithm to determine the timing of spikes with an order of magnitude greater precision than other techniques barring stimulus information, and arbitrarily high precision with the stimulus information.  Furthermore, all the parameters of the neuron, including those governing the calcium parameters, can be fit with a reasonable amount of data.

%More specifically, we assume the dynamics of the observable neurons are governed by a generalized linear model (GLM).  Models of this form are quickly gaining favor in the field of computational neuroscience for a variety of reasons\cite{McCullaghNelder89}: (i) they are relatively simple, (ii) they can be considered generalizations of a leaky-integrate-and-fire or spike response model, (iii) they have no local extrema, enabling fast parameter estimation and stimulus identification, and (iv) they have previously been shown to accurately account for neural dynamics in a wide range of experimental substrates and conditions.  On top of the neuron model, the calcium concentration jumps after each spike and then decays back down.  Observations are both noisy and intermittent.  The model is constructed to be a nonlinear, discrete-time, state-space model.  A Sequential-Monte-Carlo Expectation-Maximization (SMC-EM) algorithm is a natural framework for solving both the abovementioned problems.  The SMC element is required because the model is nonlinear and non-Gaussian.  The EM element is required because the true calcium concentrations and spike times are \emph{hidden}.  A couple different implementations developed trade off between conceptual simplicity and computational efficiency.  Regardless, with only noisy and intermittent calcium observations, one can both infer the precise timing of spikes and estimate the parameters governing the model.

%The methods section begins with a description of the assumed model system, followed by a brief general overview of SMC-EM algorithms, and then details for our model.  The results first discuss the ability to infer the precise spike trains, with an up to ten-fold improvement in temporal precision over standard techniques, and then demonstrate the accuracy of the parameter estimates.  Finally, we mention several potential pitfalls and extensions in the discussion. 