%\documentclass{article}
%\usepackage{amsmath}
%\usepackage{a4wide}
%\providecommand{\ve}[1]{\boldsymbol{#1}}
%\newcommand{\thetn}{\ve{\theta}}
%\newcommand{\p}{P_{\thetn}}
%\title{k-mixture Conditional Sampler}
%\begin{document}
%\maketitle

More efficient sampling can be achieved by using a sampling distribution that explicitly considers the observations.  In particular, one can sample conditioned on the \emph{next} observation.  Thus, if $u$ is the time of the previous observation, and $v$ is the time of the next observation, for all $t \in(u,v]$, one would like to define $\q$ as $\p[\ve{H}_t|\ve{H}_{t-1}^{(i)},O_v]$, which may be expanded:

\begin{multline} \label{aeq:XO_opt}
%\begin{split}
\p[\ve{H}_t|\ve{H}_{t-1}^{(i)},O_v] = \p [\{\ve{h},n,C\}_t | \{\ve{h},n,C\}_{t-1}^{(i)}, F_v]
\\ = \frac{1}{Z} \p \big[ \ve{h}_t | \ve{h}^{(i)}_{t-1}, n_{t-1}^{(i)} \big] \p \big[ n_t | \ve{h}^{(i)}_{t} \big] \p \big[ C_t | C^{(i)}_{t-1}, n_t^{(i)} \big] \p \big[ F_v | C_t \big]
%\end{split}
\end{multline}

\noindent where the only difference between \eqref{eq:XO_opt} and \eqref{eq:NO_opt} is the presence of the future observation probability,  $\p[F_v | C_t]$.  Thus, this future observation probability must be computed for all $t \in(u,v]$.  The idea for computing this future observation probability is depicted in Figure \ref{fig:pf}\textbf{(B)}.  At time $v$ (the time of the next observation), we have a Gaussian distribution governing where the calcium may be, centered at the observation value (this follows from \eqref{eq:F_t}) (black line at $v$, $+$ marks the mean). In this case

\begin{align}
\p[F_v | C_t] = \mathcal{N}[C_t; F_v, \sigma_F^2]
\end{align}

\noindent which follows from \eqref{eq:F_t} and the fact that $\mathcal{N}[x;\mu,\sigma^2]=\mathcal{N}[\mu;x,\sigma^2]$ (see Proposition \ref{prop:switch} for proof).  At $v-1$, the neuron could either have spiked or not.  If the neuron did not spike, for the calcium to be where it is at time $v$, the calcium should do the inverse of decay (remember that the recursion goes backwards).  However, if the neuron did spike, the calcium should be $\beta$ \emph{below} its value at time $v$.  In either case, because the noise on the calcium transitions is Gaussian, the distribution maintains its Gaussianity, and slightly increases its variance.  Thus, at time $v-1$, the distribution of calcium is a \emph{mixture of Gaussians}.  At $v-1$, we have a \emph{2-component mixture}, one component for $n_{v-1}=1$ and one for $n_{v-1}=0$ (black line at $v-1$, $+$'s mark the means for the 2 components). The component coefficient (probability of being in that component), $a_{n,v-1}$ is simply the expected probability of spiking or not, $E[n_{v-1}]$. When no spike history terms are present, this expectation may be computed exactly.  For instance, the expected probability of spiking at time $v-1$ is

\begin{align} \label{aeq:E[a_1]}
a_{1,v-t} = 1-e^{-f(b+\ve{k}'\ve{x}_{v-1})dt}
\end{align}

\noindent and the expected probability of not spiking is simply $a_{0,v-t}=1-a_{1,v-t}$.  When spike history terms are present, $f(\cdot)$ would also be a function of $\ve{h}_{v-1}$, which has not yet been sampled (because $v-1>t$).  We therefore must recursively approximate the expected value for each spike history term using

\begin{align} \label{aeq:E[h]}
%\begin{split}
E[h_{lt}] &= E[d_{hl} h_{l,t-1} + n_{t-1} + \varepsilon_{ht}] = d_{hl} E[h_{l,t-1}] + E[n_{t-1}]
%\end{split}
\end{align}

\noindent for all $t\in(u,v)$. Then, we let

\begin{align} \label{aeq:E[a_1h]}
a_{1,t} = E[n_{t}=1] \approx 1-e^{-f(b+\ve{k}'\ve{x}_t + \ve{\omega}'E[\ve{h}_t])dt}
\end{align}

\noindent and $a_{0,t}=1-a_{1,t}$. By iterating between \eqref{eq:E[h]} and \eqref{eq:E[a_1h]} for $t\in(u,v)$, we get the expected probability of the neuron spiking at any time.  We may then plug $a_{n,v-1}$ into the backward step:

\begin{align} \label{aeq:mix1}
\p[F_v | C_{v-1}] = \sum_{n=0,1} a_{n,v-1} \int \p[F_v | C_v] \p[C_v | C_{v-1}, n_v=n] dC_v
\end{align}

\noindent where

\begin{align}
\p[C_t | C_{t-1}, n_t=n] =
\begin{cases}
\mathcal{N}[C_t; d_c C_{t-1} + \beta n_t, \sigma_c^2 dt] & \text{if } n_t=1\\
\mathcal{N}[C_t; d_c C_{t-1}, \sigma_c^2 dt] & \text{if } n_t=0
\end{cases}
\end{align}

\noindent Evaluating this integral using Proposition \ref{prop:uni_gauss_prod} and Corollaries \ref{cor:z} and \ref{cor:z2} yields

\begin{align}
\p[F_v | C_{v-1}] = \sum_{n=0,1} a_n \mathcal{N}\left[C_{v-1};\frac{F_v - \beta n}{d_c}, \frac{\sigma_F^2 + \sigma_c^2 dt}{d_c^2}\right]
\end{align}

\noindent (see Corollary \ref{cor:bact} for derivation). Recursing backward one more step yields a 4-component mixture, as each component in the mixture at $v-1$ could have gotten there either from the neuron spiking or not at time $v-2$ (black line and $+$'s at $v-2$).  The coefficient for each of the 4 components is proportional to the expected probability of having that particular \emph{sequence} of spikes, i.e., at $v-2$, we have 4 possible sequences: $(00)$, $(01)$, $(10)$, and $(11)$ corresponding to no spikes, only spiking at time $v-1$, only spiking at time $v-2$ and spiking at both $v-1$ and $v-2$, respectively.  This suggests that at time $v-t$, there will be a $2^{v-t}$-component mixture, where each component may be indexed by a binary number of length $v-t$. %Thus, introducing the notation $\mathcal{M}[C_{t}; \ve{a}_{t}, \ve{\mu}_{t}, \sigma_{t}^2]$ to indicate the mixture of Gaussians, each of which is a function of $C_{t}$, with component coefficients $\ve{a}_t=[a_1, a_M]$, means $\ve{\mu}_{t}=[\mu_{1,t},\mu_{M,t}]$ and variance $\sigma_{m,t}^2$.  Note that $M$ is the number of components in the mixture, which for this model, equals $2^{v-t}$.  Further note that the variance of all the components of the mixture are identical, as it is independent of whether the neuron spikes.
Formally, we can apply this intuition to perform a backwards recursion for $\p[F_v | C_t]$:

\begin{align}
\p[F_v | C_{t-1}] = \sum_{n=0,1} a_n
\sum_{m=1}^{2^{v-t}} a_{ms} \mathcal{N}\left[C_{t-1}; \frac{\mu_{ms} - \beta n}{d_c}, \frac{\sigma_t^2 + \sigma_c^2 dt}{d_c^2}\right] %= \sum_m^{2^{v-(t-1)}} a_m \mathcal{N}[C_{t-1}; \mu_{ms}, \sigma_t^2]
\end{align}

\noindent where $m$ indexes one of the $2^{v-t}$ possible spike trains between $t$ and $v$, corresponding to one component of the mixture.  Thus, $a_{ms}$ and $\mu_{ms}$ indicate the component coefficient and mean, respectively, from time $t$, and $\sigma_t^2$ indicates the variance for all the components at time $t$ (they are all equal because they are not a function of whether the neuron spikes).
%Thus, letting $\mathcal{M}[F_v | C_t, n_{t:v}]$ be the mixture representing $\p[F_v | C_t]$, we can compute it recursively using
%\begin{align}
%\p \big[ F_v | C_t \big] = \mathcal{M}[F_v | C_t, n_{t:v}] = \sum_m^{2^{v-t}} a_m \mathcal{N}[C_t; \mu_{ms}, \sigma_{t}^2]
%\end{align}
%\noindent where $m$ is the $v-t$-bit binary number indexing each component of the mixture, $a_m$ is the \emph{component weight}, $\mu_{ms}$ is the mean of component $m$ at time $t$, and $\sigma_t^2$ is the variance of each component at time $t$ (they all have equal variance because the calcium noise is independent of everything else).

An interesting observation is that at $v-2$, two of the means seem to be completely overlapping.  In fact, those two components correspond to $(01)$ and $(10)$, i.e., the sequences with exactly one spike.  This follows from the fact that the calcium time constant is much larger than the step size, $\tau_c \gg dt$; therefore, the amount of decay (or rather, inverse decay) in a few time steps is essentially negligible.  One can therefore approximate the two components corresponding to a single spike at $v-2$ into one component by using the following rule:

%\begin{align} \label{aeq:mix_approx1}
%\sum_m a_{ms} \mathcal{N}[C_t; \mu_{ms}, \sigma_t^2] &\approx a_{\m t} \mathcal{N}[C_t, \mu_{\m t}, \sigma_t^2 + \sum_m a_{ms} (\mu_{ms}-\mu_{\m t})^2]
%\end{align}
\begin{align} \label{aeq:mix_approx1}
\sum_m a_{m} \mathcal{N}[C_t; \mu_{m}, \sigma^2] &\approx a_{\m} \mathcal{N}[C_t, \mu_{\m}, \sigma^2 + \sum_m a_{m} (\mu_{m}-\mu_{\m})^2]
\end{align}

\noindent where $a_\m=\sum_m a_m$ and $\mu_{\m}=\sum_m a_{m} \mu_{m}$.  More generally, at any time $v-t$, all the components resulting from the same number of spikes between $t$ and $v$ can be combined into a single component.  One must simply take care to modify the component weights, means, and variances using \eqref{eq:mix_approx1}.  Upon doing so, at time $t$, instead of a mixture with $2^{v-t}$ components, we are left with a mixture of $v-t+1$ components (i.e., one component per possible number of spikes until time $v$):
%
%\begin{align}
%\p \big[ F_v | C_t \big] \approx \mathcal{M}^\ast[F_v | C_t, n_{t:v}] = \sum_{\m}^{v-t+1} a_{\m} \mathcal{N}[C_t; \mu_{\m t}, \sigma_{\m t}^2]
%\end{align}
%
%\noindent where $\m$ indexes each of the components.
Note that upon making this approximation, each component has a different variance, which should be clear from \eqref{eq:mix_approx1} (each $\m$ component combines different $m$ components).  This approximation drastically reduces the computational load of using the conditional sampler.  For instance, imagine that the frame rate, $f$, is $10$ Hz and the step size, $dt$, is $2$ msec.  In that scenario, without the approximation, there would be up to $v-u=1/(f dt)=50$ time steps between observations, yielding a $2^{50}$ component mixture in the no-approximation situation, versus a $51$ component mixture when using our approximation!  As such, we use this approximation for $\p[F_v | C_t]$ when sampling.  Below, we indicate how one can sample from the conditional sampler for each hidden state, $\ve{h}_t$, $n_t$, and $C_t$.

\paragraph{Sampling Spike Histories}

While one could sample the spike histories conditioned on the observations, because they are functions of $n_{t-1}$, they can simply be sampled from their transition distributions without much loss of efficiency.  Therefore, for each spike history term, we sample from $\p[h_{ls} | h_{l,t-1}]$, which is given by \eqref{eq:samp_h}.

\paragraph{Sampling Spikes}

To sample spikes from \eqref{eq:XO_opt}, one must integrate out the other hidden states, to get a sampling distribution which is a function only of $n_t$.  In particular, we determine $q[n_t]$, the sampling distribution for $n_t$, by integrating \eqref{eq:XO_opt} with respect to $\ve{h}_t$ and $C_t$:

\begin{align}
\nonumber q[n_t] &\sim \frac{1}{Z} \int d\ve{h}_t  \p \big[ n_t | \ve{h}_t \big] \p \big[ \ve{h}_t | \ve{h}^{(i)}_{t-1}, n_{t-1}^{(i)} \big] \int  \p \big[ C_t | C^{(i)}_{t-1}, n_t \big] \p[F_v|C_t] dC_t
\\&=\frac{1}{Z} \p \big[ n_t | \ve{h}^{(i)}_t \big] \int \p \big[ C_t | C^{(i)}_{t-1}, n_t \big] \p[F_v | C_t] dC_t
\end{align}

\noindent where the equality follows from the fact that $\ve{h}_t$ was already sampled. One can compute the above integral using the fact that the integral of a product of Gaussians is again a Gaussian, yielding a Gaussian for each component in the mixture:

\begin{align} \label{aeq:Z_t}
\mathcal{N}_{\m}^{(i)}[n_t, F_v] =
\frac{1}{\sqrt{2 \pi (\sigma_{\m t}^2 + \sigma_c^2 dt)}} \exp \left\{-\frac{1}{2}\left(\frac{\mu_{\m t}-d_c C_{t-1}^{(i)} + \beta n_t)} {\sigma_{\m t}^2 + \sigma_c^2 dt}\right)^2\right\}
\end{align}

%\begin{align} \label{aeq:Z_t}
%\mathcal{N}_{\m}^{(i)}[n_t;  F_v] =
%\frac{1}{\sqrt{2 \pi (\sigma_{\m t}^2 + \sigma_c^2 dt)}} \exp \left\{-\frac{1}{2}\left(\frac{\mu_{\m t}-((1-dt/\tau_c) C_{t-1}^{(i)} + \beta n_t)} {\sigma_{\m t}^2 + \sigma_c^2 dt}\right)^2\right\}
%\end{align}

\noindent which we compute for $n_t=0$ and $n_t=1$ (see Corollary \ref{cor:bact2}  for a derivation).  Thus, for each particle, one samples from

\begin{align} \label{aeq:qn}
\widehat{q}[n_t]&=\mathcal{B}[n_t; 1-e^{-f(y_t^{(i)}) dt}] \sum_{\m=0}^{v-t} \mathcal{N}_{\m}^{(i)}[n_t, F_v]\\
q[n_t]&=\frac{\widehat{q}[n_t]}{\sum_{n_t=0,1} \widehat{q}[n_t]}
\end{align}

\paragraph{Sampling Calcium}

In a similar fashion as done to generate $q[n_t]$, one can marginalize $\ve{h}_t$ and \emph{$n_t$} out of \eqref{eq:XO_opt}, yielding:

\begin{equation} \label{aeq:qC}
\begin{split}
C_t^{(i)} \sim q[C_t] &= \frac{1}{Z}  \p[F_v | C_t] \p[C_t | C_{t-1}^{(i)}, n_t^{(i)}]
\\&= \frac{1}{Z}  \sum_{\m=n_t^{(i)}}^{v-t} a_{\m t} \mathcal{N}[C_t; \mu_{\m t}, \sigma_{\m t}^2] \mathcal{N}[C_t; d_c C_{t-1}^{(i)} + \beta n_t^{(i)}, \sigma_c^2 dt]
\\ &= \frac{1}{Z}  \sum_{\m=n_t^{(i)}}^{v-t} a_{\m t} \mathcal{N}\left[C_t; \mu_{\m t} + d_c C_{t-1}^{(i)} + \beta n_t^{(i)}, \sigma_{c\m t}^2 \right]
\end{split}
\end{equation}

\noindent which also follows from Proposition \ref{prop:uni_gauss_prod} and Corollary \ref{cor:z} (and is derived in Corollary \ref{cor:bact3}), and

\begin{align}
\sigma_{c\m t}^{-2} &= \sigma_{\m t}^{-2} + (\sigma_c dt)^{-2}\\
\mu_{c\m t}^{(i)} &=\sigma_{c\m t}^2 \left( \frac{\mu_{\m t}} {\sigma_{\m t}^2} + \frac{(1-dt/\tau_c) C_{t-1}^{(i)} + \beta n_t^{(i)}}{\sigma_c^2 dt}\right)
\end{align}

To sample from this mixture, one first samples a component according to $a_\m$, and then samples from the Gaussian corresponding to that component.  Note, however, that the sum in \eqref{eq:qC} starts at $n_t^{(i)}$, because if $n_t^{(i)}=1$, then the component corresponding to \emph{zero} spikes between $t$ and $v$ should not be considered for that particle.

\paragraph{Computing the Weights}

At each time step, the weights are updated according to \eqref{eq:SIS}.  Expanding \eqref{eq:SISa} for the conditional sampler results in

\begin{align} \label{aeq:cond_w}
\widetilde{w}_t^{(i)} &= \frac{\p\big[O_t | \ve{H}_t^{(i)}\big]
\p\big[n_t^{(i)} | \ve{h}_{t-1}^{(i)}\big] \p\big[C_t^{(i)} | C_{t-1}^{(i)}, n_t^{(i)}\big] w_{t-1}^{(i)}}{q\big[n_t^{(i)}\big] q\big[C_t^{(i)}\big]}
\end{align}

\noindent because $\p[\ve{h}_t^{(i)} | \ve{h}_{t-1}^{(i)}]$ cancels from the numerator and denominator.  At observation times, one resamples if $\widehat{N_{eff}}$ is less than some threshold (usually taken to be $N/2$), where $\widehat{N_{eff}}$ is given by

\begin{align}
\widehat{N_{eff}}^{-1} = \sum_{i=1}^N w_t^{(i)}
\end{align}

\noindent which indicates whether too much of the weight is centered on too few particles\cite{Rubin88}.  Table \ref{tab:back} provides pseudocode for implementing the conditional particle filter, which can act as the forward step in the SMC-EM algorithm.

\begin{table}[h]
\caption{Pseudocode for Conditional Sampling}
\label{tab:back}
\begin{enumerate}
\item If $t \in \mathcal{T}_o$, update the approximate observation distribution, $\mathcal{M}^\ast[F_v | C_t, n_{t:v}]$, for $t \in (u,v]$.
\item For each particle, $i\in[1, N]$, at \emph{every} time step, update each state:
    \begin{enumerate}
    \item For $l\in [1, L]$, sample $h_{ls}^{(i)} \sim q[h_{ls}]$ using \eqref{eq:samp_h}
    \item Compute $q[n_t]$ for $n_t=0$ and $n_t=1$ using \eqref{eq:qn} and then sample $n_t^{(i)} \sim q[n_t]$
    \item Compute $q[C_t]$ using \eqref{eq:qC} then sample $C_t^{(i)} \sim q[C_t]$
    \end{enumerate}
\item For each particle, $i=\in[1,N]$, at every time step, update weights, $w_t^{(i)}$, using \eqref{eq:cond_w}
\item At every \emph{observation} time step, $t \in \mathcal{T}_o$, stratified resample if $\widehat{N_{eff}}<N/2$.
\end{enumerate}
\end{table}

%\end{document} 